\documentclass[conference]{IEEEtran}
\renewcommand\IEEEkeywordsname{Słowa kluczowe}
\IEEEoverridecommandlockouts
% The preceding line is only needed to identify funding in the first footnote. If that is unneeded, please comment it out.
\usepackage[utf8]{inputenc}
\usepackage[T1]{fontenc}
\usepackage{cite}
\usepackage{amsmath,amssymb,amsfonts}
\usepackage{algorithmic}
\usepackage{graphicx}
\usepackage{textcomp}
\usepackage{xcolor}
\usepackage[polish]{babel}
\usepackage{hyperref}
\def\BibTeX{{\rm B\kern-.05em{\sc i\kern-.025em b}\kern-.08em
    T\kern-.1667em\lower.7ex\hbox{E}\kern-.125emX}}
\usepackage[backend=biber,
        style=ieee,     % styl numeryczny IEEE prawie jak PN
        sorting=nyt,    % sortowanie spisu po nazwiskach
        citestyle=numeric-comp % kompaktowe odnośniki numeryczne typu [21-23]
    ]{biblatex}
\addbibresource{article.bib}
\begin{document}

\title{Porównanie algorytmów kryptografii asymetrycznej i zastosowania}

\author{\IEEEauthorblockN{1\textsuperscript{st} Krzysztof Dąbrowski}
\IEEEauthorblockA{\textit{Wydział Elektryczny} \\
\textit{Politechnika Warszawska}\\
Warszawa, Polska \\
krzysztof.dabrowski7.stud@pw.edu.pl}
\and
\IEEEauthorblockN{2\textsuperscript{nd} Hussein Hazime}
\IEEEauthorblockA{\textit{Wydział Elektryczny} \\
\textit{Politechnika Warszawska}\\
Warszawa, Polska \\
01202345@pw.edu.pl}
\and
\IEEEauthorblockN{3\textsuperscript{rd} Krzysztof Rudnik}
\IEEEauthorblockA{\textit{Wydział Elektryczny} \\
\textit{Politechnika Warszawska}\\
Warszawa, Polska \\
01151681@pw.edu.pl}
\and
\IEEEauthorblockN{4\textsuperscript{th} Piotr Szczerba}
\IEEEauthorblockA{\textit{Wydział Elektryczny} \\
\textit{Politechnika Warszawska}\\
Warszawa, Polska \\
01159503@pw.edu.pl}
\and
\IEEEauthorblockN{5\textsuperscript{th} Jakub Więcław}
\IEEEauthorblockA{\textit{Wydział Elektryczny} \\
\textit{Politechnika Warszawska}\\
Warszawa, Polska \\
01159706@pw.edu.pl}
% \and
% \IEEEauthorblockN{6\textsuperscript{th} Given Name Surname}
% \IEEEauthorblockA{\textit{dept. name of organization (of Aff.)} \\
% \textit{name of organization (of Aff.)}\\
% City, Country \\
% email address or ORCID}
}

\maketitle


\begin{abstract}
% This document is a model and instructions for \LaTeX.
% This and the IEEEtran.cls file define the components of your paper [title, text, heads, etc.]. *CRITICAL: Do Not Use Symbols, Special Characters, Footnotes,
% or Math in Paper Title or Abstract.
% Przedstawienie wybranych algorytmów kryptografii asymetrycznej. Przegląd zastosowań z omówieniem rozwiązań korzystających z omawianych algorytmów. Porównanie bezpieczeństwa wybranych algorytmów. Omówienie możliwych przyszłych trendów dotyczących kryptografii asymetrycznej.

%do zredagowania
Artykuł przedstawia wybrane algorytmy kryptografii asymetrycznej, omawiając ich zasadę działania oraz zastosowania w praktycznych systemach zabezpieczeń. Dokonano porównania bezpieczeństwa poszczególnych metod. Przegląd uwzględnia również współczesne rozwiązania wykorzystujące te algorytmy oraz potencjalne kierunki ich rozwoju w kontekście rosnących zagrożeń, takich jak komputery kwantowe. Artykuł dostarcza czytelnikowi kompleksowego spojrzenia na obecny stan oraz przyszłość kryptografii asymetrycznej.

% Ten artykuł przedstawia informacje, które udało się zgromadzić naszemu zespołowi na temat algorytmów kryptografii asymetrycznej. Przedstawione są w nim podstawy matematyczne algorytmów RSA, ECC oraz Puzli Merkele'a. Zespół dokonuje porównania metod działania oraz zastosowań współczesnych algorytmów kryptografii asymetrycznej RSA, a także ECC.
\end{abstract}

\begin{IEEEkeywords}
% component, formatting, style, styling, inser
algorytmy, RSA, ECC, kryptografia asymetryczna
\end{IEEEkeywords}

\section{Wstęp}
% This document is a model and instructions for \LaTeX.
% Please observe the conference page limits.


Współczesna komunikacja elektroniczna wymaga zabezpieczeń, które pozwolą na zachowanie poufności i~integralności przesyłanych informacji. Szyfrowanie jest podstawowym mechanizmem służącym do osiągnięcia tego celu, a kryptografia asymetryczna odgrywa w tym szczególnie ważną rolę. Wykorzystuje ona parę kluczy: publiczny do szyfrowania i prywatny do odszyfrowywania, co pozwala na efektywne zabezpieczenie danych przesyłanych pomiędzy różnymi podmiotami.

W artykule tym zostaną przedstawione podstawy teoretyczne kryptografii asymetrycznej, w tym algorytmy RSA i~ECC, oraz ich zastosowania w praktyce. Omówiona będzie też przyszłość kryptografii asymetrycznej szczególnie w odniesieniu do rozwoju komputerów kwantowych.


\section{Matematyczna teoria algorytmów kryptografii asymetrycznej}

\subsection{RSA}
\begin{itemize}
    \item Wybieramy dwie duże liczby pierwsze \( p \) i \( q \).
    \item Obliczamy \( n = p \cdot q \), oraz funkcję Eulera \( \varphi(n) = (p - 1)(q - 1) \).
    \item Wybieramy liczbę \( e \) (klucz publiczny), taką że:
    \[
    \text{gcd}(e, \varphi(n)) = 1
    \]
    \item Obliczamy liczbę \( d \) (klucz prywatny), tak aby:
    \[
    e \cdot d \equiv 1 \mod \varphi(n)
    \]
      \begin{itemize}
        \item Klucz publiczny: \( (e, n) \)
        \item Klucz prywatny: \( (d, n) \)

    \end{itemize}


     \item  Jeśli wiadomość \( M \) (gdzie \( M < n \)) ma zostać zaszyfrowana, obliczamy:
    \[
    C = M^e \mod n
    \]
    gdzie \( C \) to zaszyfrowana wiadomość.

     \item  Odbiorca używa klucza prywatnego \( d \) i oblicza:
    \[
    M = C^d \mod n
    \]
    \end{itemize}
\subsection{ECC}

Kryptografia krzywych eliptycznych (ECC) jest definiowana za pomocą następującego ogólnego równania w dwóch zmiennych z współczynnikami:

\[
    y^2 = x^3 + ax + b
\]

gdzie $a$ i $b$ są współczynnikami krzywej eliptycznej.

Delta krzywej eliptycznej jest określony jako:

\[
    \Delta = 4a^3 + 27b^2 \neq 0
\]

Warunek $\Delta \neq 0$ zapewnia, że krzywa tworzy \textbf{grupę algebraiczną}, co jest niezbędne do zastosowania jej w kryptografii. Jeśli $\Delta = 0$, struktura matematyczna krzywej nie nadaje się do użycia w szyfrowaniu.

 \textbf{Globalne elementy publiczne:}
    \begin{itemize}
        \item Wybór krzywej eliptycznej $E_q(a, b)$
        \item Wybór punktu bazowego $G(x, y)$ o dużym rzędzie $n$.
    \end{itemize}

    \textbf{Generowanie klucza Użytkownika A:}
    \begin{itemize}
        \item Użytkownik A wybiera klucz prywatny $V_A$, gdzie $V_A < n$.
        \item Oblicza klucz publiczny $P_A(x, y) = V_A \times G(x, y)$.
    \end{itemize}

    \textbf{Generowanie klucza Użytkownika B:}
    \begin{itemize}
        \item Użytkownik B wybiera klucz prywatny $V_B$, gdzie $V_B < n$.
        \item Oblicza klucz publiczny $P_B(x, y) = V_B \times G(x, y)$.
    \end{itemize}



    \begin{itemize}
        \item Użytkownik A wybiera wiadomość $P_m(x, y)$ oraz losową liczbę $k$, gdzie $1 < k < q$.
        \item Tworzy szyfrogram $C_m = ((k \times G(x, y)), (P_m(x, y) + k \times P_B(x, y)))$.
    \end{itemize}
    \begin{itemize}
        \item Użytkownik B otrzymuje szyfrogram $C_m$ jako $((x, y), (x', y'))$.
        \item Odszyfrowuje wiadomość:
        \[
            P_m(x, y) = (P_m(x, y) + k \times P_B(x, y)) - (k \times V_B \times G(x, y))
        \]
        \item Po wykonaniu operacji pozostaje oryginalna wiadomość $P_m(x, y)$.
    \end{itemize}

\section{Praktyczne przykłady zastosowania algorytmów kryptograficznych}
\subsection{Podpis cyfrowy}
Podpis cyfrowy zapewnia integralność, uwierzytelnianie oraz niezaprzeczalność wiadomości. Algorytm podpisu cyfrowego RSA opiera się na szyfrowaniu skrótu wiadomości kluczem prywatnym, natomiast ECC wykorzystuje schemat ECDSA.
\subsection{Certyfikaty}
Certyfikaty cyfrowe X.509 są wykorzystywane w protokołach TLS i HTTPS do uwierzytelniania stron internetowych oraz zapewnienia bezpiecznej komunikacji. Jeśli certyfikat został wystawiony przez zaufaną instytucję to może być to wyznacznik tego, że strona jest zaufana.
\subsection{Kryptowaluty}
Bitcoin i inne kryptowaluty wykorzystują ECC (secp256k1) do generowania adresów i podpisywania transakcji, zapewniając bezpieczeństwo operacji finansowych.
\subsection{Komunikatory ,,End to End''}
Ideą komunikatorów ,,End to End'' jest to, aby przy użyciu kryptografii asymetrycznej sprawić żeby tylko uczestnicy danej rozmowy mieli do niej dostęp. Na serwerach komunikatorów rozmowa przechowywana jest w formie zaszyfrowanej, a członkowie rozmowy przy pomocy swoich kluczy mogą pobrać wiadomości i je rozszyfrować na swoim urządzeniu.
\subsection{Internet rzeczy}
Urządzenie IoT muszą dobrze szyfrować swoją komunikację, gdyż
\begin{itemize}
    \item dostęp do komunikacji i włączenie się do niej nieuprawnionemu urządzeniu daje dostęp, który potencjalnie może wywołać dużo szkód,
    \item przesyłają one dane wrażliwe (takie jak obraz kamer monitoringu czy dane z urządzeń medycznych) i nieuprawniony dostęp do tych danych jest poważnym naruszeniem prywatności i bezpieczeństwa.
\end{itemize}

\section{Porównywanie bezpieczeństwa}

\subsection{Metoda porównania}
Bezpieczeństwo algorytmów kryptograficznych można mierzyć za pomocą liczby bitów bezpieczeństwa \cite{BitSecurityOfCryptographicPrimitives}. Określa się ją jako \( x = \log_2(N) \), gdzie \( N \) to średnia liczba operacji koniecznych do złamania szyfru.

\subsection{Bity bezpieczeństwa RSA}

Przy atakach na RSA wykorzystuje się rozkład na czynniki pierwsze, w celu pozyskania klucza prywatnego z klucza publicznego. Najszybszym klasycznym algorytmem \footnote{tz. nie korzystający z matematyki kwantowej} faktoryzacji liczb dużych liczb jest ogólne sito ciała liczbowego (GNFS\footnote{ang. General Number Field Sieve}), którego złożoność wyraża się wzorem \cite{GNFSImplementation}:

\[
L(n) = \exp\left(\left(\frac{64}{9}\right)^{1/3} (\ln n)^{1/3} (\ln \ln n)^{2/3}\right)
\]

Liczba bitów bezpieczeństwa wynosi \( \log_2(L(n)) \).

\subsection{Bity bezpieczeństwa ECC}

W przypadku ECC głównym matematycznym mechanizmem zapewniającym bezpieczeństwo jest problem logarytmów dyskretnych na krzywych eliptycznych (ECDLP\footnote{ang. Elliptic Curve Discrete Logarithm Problem})
Najwydajniejszy klasyczny algortym rozwiązania problemu ECDLP jest algorytm rho Pollarda \cite{SolvingECDLP}, który dla przestrzeni wielkości \( k \) wymaga \(\sqrt{k}\) kroków.
Przekłada się to na to, że aby osiągnąć \( x \) bitów bezpieczeństwa, potrzebny jest klucz o wielkości \( 2x \).

W praktyce jednak złożoność tego algorytmu nie zależy bezpośrednio od wielkości dyskretnej przestrzeni, lecz od ilości dyskretnych punktów krzywej, których jest zazwyczaj mniej.
Rzeczywiste bezpieczeństwo można szacować jako \( \approx 0.886 \times \sqrt{k} \). Konkretny współczynnik jest zależny od parametrów użytej krzywej.

\subsection{Porównanie długości klucza}
Do osiągnięcie danej liczby bitów bezpieczeństwa jest wymagany znacznie dłuższy klucz przy użyciu algorytmu RSA niż w przypadku ECC.

\begin{table}[h]
    \centering
    \begin{tabular}{|c|c|c|}
        \hline
        \textbf{Bity bezpieczeńśtwa} & \textbf{Długość klucza RSA} & \textbf{Długość klucza ECC} \\
        \hline
        75  & 1024  & 160  \\
        100 & 2048  & 224  \\
        125 & 3072  & 256  \\
        150 & 7680  & 384  \\
        175 & 15360 & 512  \\
        \hline
    \end{tabular}
    \caption{Porównanie długości kluczy RSA i ECC dla różnych poziomów bezpieczeństwa}
    \label{tab:poziomy_bezpieczenstwa}
\end{table}

Obserwacja ta nie oznacza, że algorytm RSA jest mniej bezpieczny niż ECC, lecz uwypukla konieczność stosowania długich kluczy przy zastosowaniu tego algorytmu.

\subsection{Podatne krzywe ECC}

Przy zastosowaniu ECC niezwykle istotne jest, żeby problem ECDLP był możliwe trudny.
Jednak nie wszystkie krzywe to gwarantują. Ataki na słabe krzywe obejmują \cite{WeakCurvesInEllipticCurveCryptography} algorytm Pohlinga-Hellmana i Smarta, które są znacznie bardziej wydajne niż algorytm rho Pollarda. Ich zastosowanie wymaga spełnienia przez krzywe konkretnych warunków.

\section{Przyszłość kryptografii asymetrycznej}
\title{Elliptic Curve Cryptography vs. RSA: Podatność na Algorytm Shora}
\author{}
\date{\today}

\begin{document}

\maketitle


Rozwój komputerów kwantowych stanowi poważne zagrożenie dla obecnie stosowanych systemów kryptograficznych. Jednym z najbardziej znanych algorytmów zdolnych do ich przełamania jest algorytm Shora, który pozwala na efektywne faktoryzowanie liczb oraz rozwiązywanie problemu logarytmu dyskretnego. W tym artykule przeanalizujemy, dlaczego kryptografia oparta na krzywych eliptycznych (ECC) jest bardziej podatna na atak algorytmem Shora niż RSA, pomimo jej lepszej odporności na klasyczne ataki.

\subsection{Algorytm Shora i jego znaczenie}
Algorytm Shora to kwantowy algorytm wielomianowy, który rozwiązuje dwa fundamentalne problemy kryptograficzne:
\begin{itemize}
\item \textbf{Faktoryzacja liczb} (RSA)
\item \textbf{Problem logarytmu dyskretnego} (np. ECC)
\end{itemize}
Oba te problemy stanowią podstawę współczesnej kryptografii asymetrycznej. Klasyczne algorytmy ich rozwiązywania są niezwykle kosztowne obliczeniowo, jednak algorytm Shora pozwala na ich efektywne rozwiązanie na komputerze kwantowym.

\subsection{Porównanie odporności ECC i RSA}
Badania wykazują, że ECC jest \textbf{bardziej podatne na atak algorytmem Shora} niż RSA przy porównywalnym poziomie klasycznego bezpieczeństwa. Wpływają na to następujące czynniki:

\subsection{Liczba wymaganych kubitów}
Algorytm Shora wymaga znacznie mniej kubitów do przełamania ECC niż RSA: \cite{quantum_comp}
\begin{itemize}
\item RSA-3072 wymaga \textbf{6146 kubitów}.
\item ECC-P256 wymaga \textbf{2330 kubitów}.
\end{itemize}
Oznacza to, że komputer kwantowy zdolny do przełamania ECC może powstać znacznie wcześniej niż taki, który może przełamać RSA.

\subsection{Liczba operacji Toffoli}
Liczba bramek Toffoli (podstawowe operacje w obliczeniach kwantowych) również wskazuje na większą podatność ECC: \cite{quantum_comp}
\begin{itemize}
\item RSA-3072 wymaga \textbf{$1.86 \times 10^{13}$ operacji Toffoli}.
\item ECC-P256 wymaga \textbf{$1.26 \times 10^{11}$ operacji Toffoli} (około 100 razy mniej).
\end{itemize}
Mniejsza liczba operacji oznacza, że algorytm Shora może szybciej rozwiązać problem logarytmu dyskretnego w ECC niż rozłożyć liczbę na czynniki w RSA.

\subsubsection{Złożoność algorytmiczna}
\begin{itemize}
\item W \textbf{klasycznej kryptografii} najlepsze algorytmy łamiące ECC mają złożoność wykładniczą.
\item W przypadku RSA istnieje \textbf{subwykładniczy algorytm faktoryzacji (GNFS)}, co oznacza, że w klasycznym świecie RSA jest stosunkowo łatwiejsze do przełamania niż ECC.
\item Jednak algorytm Shora \textbf{eliminuje tę przewagę ECC}, ponieważ oba problemy rozwiązuje w czasie wielomianowym.
\end{itemize}

\subsubsection{Mniejsze klucze ECC}
NIST(National Institute of Standards and Technology) szacuje, że ECC-256 bit zapewnia podobny poziom bezpieczeństwa jak RSA-3072 bit. Oznacza to, że \textbf{przy tej samej odporności klasycznej ECC operuje na znacznie mniejszych kluczach}, co dodatkowo ułatwia przełamanie algorytmem Shora.

\subsection{Implikacje dla przyszłości kryptografii}
Ze względu na te czynniki \textbf{ECC może zostać przełamane szybciej niż RSA, gdy pojawią się komputery kwantowe zdolne do uruchomienia algorytmu Shora}. W związku z tym:
\begin{itemize}
\item NIST prowadzi standaryzację \textbf{postkwantowej kryptografii}, m.in. schematów opartych na \textbf{kratkach matematycznych (lattice-based cryptography)}, które są odporne na algorytm Shora.
\item Organizacje powinny \textbf{rozważyć migrację} na algorytmy kryptografii postkwantowej zanim stanie się to konieczne.
\end{itemize}

%Do podsumowania można dać
% \subsection{Wnioski}
% Podczas gdy ECC oferuje lepszą ochronę niż RSA przed klasycznymi atakami, okazuje się bardziej podatne na ataki kwantowe. Algorytm Shora wymaga mniej zasobów do jego złamania, co oznacza, że w przyszłości ECC może zostać przełamane szybciej niż RSA. Z tego powodu wprowadzenie kryptografii postkwantowej jest koniecznym krokiem dla bezpieczeństwa danych w nadchodzącej erze komputerów kwantowych.

% \section{Podsumowanie}
% Kryptografia w oparciu o krzywe eliptyczne jest mniej zasobożerna niż RSA przez co współczesne technologie
% \subsection{Maintaining the Integrity of the Specifications}

% The IEEEtran class file is used to format your paper and style the text. All margins,
% column widths, line spaces, and text fonts are prescribed; please do not
% alter them. You may note peculiarities. For example, the head margin
% measures proportionately more than is customary. This measurement
% and others are deliberate, using specifications that anticipate your paper
% as one part of the entire proceedings, and not as an independent document.
% Please do not revise any of the current designations.

% \section{Prepare Your Paper Before Styling}
% Before you begin to format your paper, first write and save the content as a
% separate text file. Complete all content and organizational editing before
% formatting. Please note sections \ref{AA}--\ref{SCM} below for more information on
% proofreading, spelling and grammar.

% Keep your text and graphic files separate until after the text has been
% formatted and styled. Do not number text heads---{\LaTeX} will do that
% for you.

% \subsection{Abbreviations and Acronyms}\label{AA}
% Define abbreviations and acronyms the first time they are used in the text,
% even after they have been defined in the abstract. Abbreviations such as
% IEEE, SI, MKS, CGS, ac, dc, and rms do not have to be defined. Do not use
% abbreviations in the title or heads unless they are unavoidable.

% \subsection{Units}
% \begin{itemize}
% \item Use either SI (MKS) or CGS as primary units. (SI units are encouraged.) English units may be used as secondary units (in parentheses). An exception would be the use of English units as identifiers in trade, such as ``3.5-inch disk drive''.
% \item Avoid combining SI and CGS units, such as current in amperes and magnetic field in oersteds. This often leads to confusion because equations do not balance dimensionally. If you must use mixed units, clearly state the units for each quantity that you use in an equation.
% \item Do not mix complete spellings and abbreviations of units: ``Wb/m\textsuperscript{2}'' or ``webers per square meter'', not ``webers/m\textsuperscript{2}''. Spell out units when they appear in text: ``. . . a few henries'', not ``. . . a few H''.
% \item Use a zero before decimal points: ``0.25'', not ``.25''. Use ``cm\textsuperscript{3}'', not ``cc''.)
% \end{itemize}

% \subsection{Equations}
% Number equations consecutively. To make your
% equations more compact, you may use the solidus (~/~), the exp function, or
% appropriate exponents. Italicize Roman symbols for quantities and variables,
% but not Greek symbols. Use a long dash rather than a hyphen for a minus
% sign. Punctuate equations with commas or periods when they are part of a
% sentence, as in:
% \begin{equation}
% a+b=\gamma\label{eq}
% \end{equation}

% Be sure that the
% symbols in your equation have been defined before or immediately following
% the equation. Use ``\eqref{eq}'', not ``Eq.~\eqref{eq}'' or ``equation \eqref{eq}'', except at
% the beginning of a sentence: ``Equation \eqref{eq} is . . .''

% \subsection{\LaTeX-Specific Advice}

% Please use ``soft'' (e.g., \verb|\eqref{Eq}|) cross references instead
% of ``hard'' references (e.g., \verb|(1)|). That will make it possible
% to combine sections, add equations, or change the order of figures or
% citations without having to go through the file line by line.

% Please don't use the \verb|{eqnarray}| equation environment. Use
% \verb|{align}| or \verb|{IEEEeqnarray}| instead. The \verb|{eqnarray}|
% environment leaves unsightly spaces around relation symbols.

% Please note that the \verb|{subequations}| environment in {\LaTeX}
% will increment the main equation counter even when there are no
% equation numbers displayed. If you forget that, you might write an
% article in which the equation numbers skip from (17) to (20), causing
% the copy editors to wonder if you've discovered a new method of
% counting.

% {\BibTeX} does not work by magic. It doesn't get the bibliographic
% data from thin air but from .bib files. If you use {\BibTeX} to produce a
% bibliography you must send the .bib files.

% {\LaTeX} can't read your mind. If you assign the same label to a
% subsubsection and a table, you might find that Table I has been cross
% referenced as Table IV-B3.

% {\LaTeX} does not have precognitive abilities. If you put a
% \verb|\label| command before the command that updates the counter it's
% supposed to be using, the label will pick up the last counter to be
% cross referenced instead. In particular, a \verb|\label| command
% should not go before the caption of a figure or a table.

% Do not use \verb|\nonumber| inside the \verb|{array}| environment. It
% will not stop equation numbers inside \verb|{array}| (there won't be
% any anyway) and it might stop a wanted equation number in the
% surrounding equation.

% \subsection{Some Common Mistakes}\label{SCM}
% \begin{itemize}
% \item The word ``data'' is plural, not singular.
% \item The subscript for the permeability of vacuum $\mu_{0}$, and other common scientific constants, is zero with subscript formatting, not a lowercase letter ``o''.
% \item In American English, commas, semicolons, periods, question and exclamation marks are located within quotation marks only when a complete thought or name is cited, such as a title or full quotation. When quotation marks are used, instead of a bold or italic typeface, to highlight a word or phrase, punctuation should appear outside of the quotation marks. A parenthetical phrase or statement at the end of a sentence is punctuated outside of the closing parenthesis (like this). (A parenthetical sentence is punctuated within the parentheses.)
% \item A graph within a graph is an ``inset'', not an ``insert''. The word alternatively is preferred to the word ``alternately'' (unless you really mean something that alternates).
% \item Do not use the word ``essentially'' to mean ``approximately'' or ``effectively''.
% \item In your paper title, if the words ``that uses'' can accurately replace the word ``using'', capitalize the ``u''; if not, keep using lower-cased.
% \item Be aware of the different meanings of the homophones ``affect'' and ``effect'', ``complement'' and ``compliment'', ``discreet'' and ``discrete'', ``principal'' and ``principle''.
% \item Do not confuse ``imply'' and ``infer''.
% \item The prefix ``non'' is not a word; it should be joined to the word it modifies, usually without a hyphen.
% \item There is no period after the ``et'' in the Latin abbreviation ``et al.''.
% \item The abbreviation ``i.e.'' means ``that is'', and the abbreviation ``e.g.'' means ``for example''.
% \end{itemize}
% An excellent style manual for science writers is \cite{b7}.

% \subsection{Authors and Affiliations}
% \textbf{The class file is designed for, but not limited to, six authors.} A
% minimum of one author is required for all conference articles. Author names
% should be listed starting from left to right and then moving down to the
% next line. This is the author sequence that will be used in future citations
% and by indexing services. Names should not be listed in columns nor group by
% affiliation. Please keep your affiliations as succinct as possible (for
% example, do not differentiate among departments of the same organization).

% \subsection{Identify the Headings}
% Headings, or heads, are organizational devices that guide the reader through
% your paper. There are two types: component heads and text heads.

% Component heads identify the different components of your paper and are not
% topically subordinate to each other. Examples include Acknowledgments and
% References and, for these, the correct style to use is ``Heading 5''. Use
% ``figure caption'' for your Figure captions, and ``table head'' for your
% table title. Run-in heads, such as ``Abstract'', will require you to apply a
% style (in this case, italic) in addition to the style provided by the drop
% down menu to differentiate the head from the text.

% Text heads organize the topics on a relational, hierarchical basis. For
% example, the paper title is the primary text head because all subsequent
% material relates and elaborates on this one topic. If there are two or more
% sub-topics, the next level head (uppercase Roman numerals) should be used
% and, conversely, if there are not at least two sub-topics, then no subheads
% should be introduced.

% \subsection{Figures and Tables}
% \paragraph{Positioning Figures and Tables} Place figures and tables at the top and
% bottom of columns. Avoid placing them in the middle of columns. Large
% figures and tables may span across both columns. Figure captions should be
% below the figures; table heads should appear above the tables. Insert
% figures and tables after they are cited in the text. Use the abbreviation
% ``Fig.~\ref{fig}'', even at the beginning of a sentence.

% \begin{table}[htbp]
% \caption{Table Type Styles}
% \begin{center}
% \begin{tabular}{|c|c|c|c|}
% \hline
% \textbf{Table}&\multicolumn{3}{|c|}{\textbf{Table Column Head}} \\
% \cline{2-4}
% \textbf{Head} & \textbf{\textit{Table column subhead}}& \textbf{\textit{Subhead}}& \textbf{\textit{Subhead}} \\
% \hline
% copy& More table copy$^{\mathrm{a}}$& &  \\
% \hline
% \multicolumn{4}{l}{$^{\mathrm{a}}$Sample of a Table footnote.}
% \end{tabular}
% \label{tab1}
% \end{center}
% \end{table}

% % \begin{figure}[htbp]
% % \centerline{\includegraphics{fig1.png}}
% % \caption{Example of a figure caption.}
% % \label{fig}
% % \end{figure}

% Figure Labels: Use 8 point Times New Roman for Figure labels. Use words
% rather than symbols or abbreviations when writing Figure axis labels to
% avoid confusing the reader. As an example, write the quantity
% ``Magnetization'', or ``Magnetization, M'', not just ``M''. If including
% units in the label, present them within parentheses. Do not label axes only
% with units. In the example, write ``Magnetization (A/m)'' or ``Magnetization
% \{A[m(1)]\}'', not just ``A/m''. Do not label axes with a ratio of
% quantities and units. For example, write ``Temperature (K)'', not
% ``Temperature/K''.

% \section*{Acknowledgment}

% The preferred spelling of the word ``acknowledgment'' in America is without
% an ``e'' after the ``g''. Avoid the stilted expression ``one of us (R. B.
% G.) thanks $\ldots$''. Instead, try ``R. B. G. thanks$\ldots$''. Put sponsor
% acknowledgments in the unnumbered footnote on the first page.

% \section*{References}

% Please number citations consecutively within brackets \cite{b1}. The
% sentence punctuation follows the bracket \cite{b2}. Refer simply to the reference
% number, as in \cite{b3}---do not use ``Ref. \cite{b3}'' or ``reference \cite{b3}'' except at
% the beginning of a sentence: ``Reference \cite{b3} was the first $\ldots$''

% Number footnotes separately in superscripts. Place the actual footnote at
% the bottom of the column in which it was cited. Do not put footnotes in the
% abstract or reference list. Use letters for table footnotes.

% Unless there are six authors or more give all authors' names; do not use
% ``et al.''. Papers that have not been published, even if they have been
% submitted for publication, should be cited as ``unpublished'' \cite{b4}. Papers
% that have been accepted for publication should be cited as ``in press'' \cite{b5}.
% Capitalize only the first word in a paper title, except for proper nouns and
% element symbols.

% For papers published in translation journals, please give the English
% citation first, followed by the original foreign-language citation \cite{b6}.

\printbibliography

% \begin{thebibliography}{99}

% \bibitem{1}
% Micciancio, D. i Walter, M. (2018, marzec). On the Bit Security of Cryptographic Primitives. s. 3-28. DOI: \href{https://doi.org/10.1007/978-3-319-78381-9_1}{10.1007/978-3-319-78381-9\_1}.\ ISBN: 978-3-319-78380-2.

% \bibitem{2}
% Balasubramanian, K. (2024, maj). Security of the Secp256k1 Elliptic Curve used in the Bitcoin Blockchain. \emph{Indian Journal of Cryptography and Network Security}, 4, s. 1-5. DOI: \href{https://doi.org/10.54105/ijcns.A1426.04010524}{10.54105/ijcns.A1426.04010524}.

% \bibitem{3}
% Puneet Gil (2019, 18 maj). Solving Elliptic Curve Discrete Logarithm Problem Using Parallelized Pollard’s Rho and Lambda Methods.

% \bibitem{4}
% Bernstein, D. i Lenstra, A. (2006, listopad). A general number field sieve implementation. \emph{Lecture Notes in Mathematics}, 1554, s. 103-126. DOI: \href{https://doi.org/10.1007/BFb0091541}{10.1007/BFb0091541}. ISBN: 978-3-540-57013-4.

% \bibitem{5}
% Bernstein, D. J. i Lange, T. \emph{Choosing safe curves for elliptic-curve cryptography}. Dostępne w: \href{https://safecurves.cr.yp.to/}{https://safecurves.cr.yp.to/}.

% \bibitem{6}
% \emph{PSGroove.com - Console Hacking 2010 Part 3 - Chaos Communication Congress}. Dostępne w: \href{https://www.youtube.com/watch?v=84WI-jSgNMQ}{https://www.youtube.com/watch?v=84WI-jSgNMQ}.

% \bibitem{7}
% Novotney, P. \emph{Weak Curves In Elliptic Curve Cryptography}.

% \bibitem{8}
% \emph{Practical Cryptography for Developers}. Dostępne w: \url{https://cryptobook.nakov.com/asymmetric-key-ciphers/elliptic-curve-cryptography-ecc#choosing-an-elliptic-curve-for-ecc}.

% \bibitem{9}
% Trail of Bits (2019, 8 lipca). \emph{Seriously, stop using RSA}. Dostępne w: \href{https://blog.trailofbits.com/2019/07/08/fuck-rsa/}{https://blog.trailofbits.com/2019/07/08/fuck-rsa/}.

% \bibitem{10}
% Gupta, N. \emph{Symmetric vs. Asymmetric Encryption – What are differences?}. Dostępne w: \href{https://www.ssl2buy.com/wiki/symmetric-vs-asymmetric-encryption-what-are-differences}{https://www.ssl2buy.com/wiki/symmetric-vs-asymmetric-encryption-what-are-differences}.

% \bibitem{11}
% Martin Roetteler, Michael Naehrig, Krysta M. Svore, and Kristin Lauter, Microsoft Research, USA \emph{Quantum Resource Estimates for Computing
% Elliptic Curve Discrete Logarithms}.

% \end{thebibliography}
% \vspace{12pt}
% \color{red}
% IEEE conference templates contain guidance text for composing and formatting conference papers. Please ensure that all template text is removed from your conference paper prior to submission to the conference. Failure to remove the template text from your paper may result in your paper not being published.

\end{document}
