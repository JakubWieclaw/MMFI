\section{Bezpieczeństwo}
% Porównanie bezpieczeństwa algorytmów RSA i ECC w zastosowaniu na aktualnymi technologiami

% Jak porównywać bezpieczeństwo algorytmów
\begin{frame}{Porównywanie bezpieczeństwa}
\textbf{Bity bezpieczeństwa \cite{BitSecurityOfCryptographicPrimitives}}
\begin{itemize}
    \pause
    \item Pojedyncza liczba
    \pause
    \item \( x = \log_2(N) \)
    \item \( x \) - liczba bitów bezpieczeństwa
    \item \( N \) - średnia ilość operacji wymaganych do złamania szyfru
\end{itemize}
\pause
\vspace{8mm}

\textit{Algorytm o sile 20 bitów bezpieczeństwa wymaga średnio $2^{20} = 1048576$ operacji do złamania.}

\end{frame}

\begin{frame}{Porówanie długości klucza}
    % Na tej samej długości klucza różne algorytmy dają różne bezpieczeństwo
\end{frame}

\begin{frame}{Problem wyboru krzywej ECC}
    % Są słabe krzywe, które można celowo wybrać jako backdoor
\end{frame}

\begin{frame}{RSA jest mocno przetestowane}
    % Są słabe krzywe, które można celowo wybrać jako backdoor
\end{frame}
