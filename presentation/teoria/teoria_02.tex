% Hussein


\begin{frame}{RSA}
    
    RSA (Rivest–Shamir–Adleman) to algorytm szyfrowania publicznego, który opiera się na trudności faktoryzacji dużych liczb.Matematyczna teoria RSA bazuje na następujących kluczowych elementach:

   1. Wybór dwóch dużych liczb pierwszych: \( p \) oraz \( q \).

   
    2. Obliczamy iloczyn tych liczb:
    \[
    n = p \times q
    \]
    Liczba \( n \) jest wykorzystywana zarówno w kluczu publicznym, jak i prywatnym.

    
    3. Funkcja Eulera dla \( n \) jest dana wzorem:
    \[
    \varphi(n) = (p-1)(q-1)
    \]

\end{frame}

\begin{frame}{RSA}
    Jest używana do obliczenia wykładnika prywatnego klucza.

    4. Wybieramy liczbę \( e \), która musi być liczbą względnie pierwszą względem \(\varphi(n)\) oraz spełnia warunek:
    \[
    \gcd(e, \varphi(n)) = 1, \quad e < \varphi(n)
    \]
    GCD: Obliczanie największego wspólnego dzielnika (GCD, greatest common divisor), oraz jest obliczany za pomocą algorytmu Euklidesa.

   
    Algorytm Euklidesa polega na powtarzaniu operacji dzielenia z resztą, aż reszta wyniesie 0.
\end{frame}

\begin{frame}{RSA}




    \begin{enumerate}
        \item Dla dwóch liczb \( a \) i \( b \), gdzie \( a \geq b \), obliczamy:
        \[
        a \div b = q \quad \text{(część całkowita)}
        \]
        \[
        r = a \mod b \quad \text{(reszta)}
        \]
        co zapisujemy jako:
        \[
        a = b \times q + r
        \]

        \item Następnie ustawiamy:
        \[
        a = b, \quad b = r
        \]
        i powtarzamy krok 1.

        \item Proces powtarzamy, aż \( b = 0 \). Gdy \( b = 0 \), wówczas:
        \[
        \gcd(a, b) = a
        \]
    \end{enumerate}



\end{frame}

\begin{frame}{RSA}
   
    5. Liczba \( d \) jest odwrotnością modularną \( e \) względem \( \varphi(n) \), tzn.:
    \[
    e \times d \equiv 1 \pmod{\varphi(n)}
    \]
   
    Należy znaleźć "modularną odwrotność" \( e \) modulo \( \varphi(n) \).

    
    W celu wyznaczenia wartości \( d \) często stosuje się \textbf{rozszerzony algorytm Euklidesa}, który nie tylko oblicza największy wspólny dzielnik (\(\gcd\)), ale również pozwala znaleźć współczynniki spełniające:

    
         \[
        e \times d + \varphi(n) \times k = \gcd(e, \varphi(n))
   \]
       
\end{frame}

\begin{frame}{RSA}
   


   
    \subsection{Klucz publiczny i prywatny}
    \begin{itemize}
        \item Klucz publiczny: \( (e, n) \)
        \item Klucz prywatny: \( (d, n) \)
    \end{itemize}

    
    6. Jeśli wiadomość \( M \) (gdzie \( M < n \)) ma zostać zaszyfrowana, obliczamy:
    \[
    C = M^e \mod n
    \]
    gdzie \( C \) to zaszyfrowana wiadomość.

   
    7. Odbiorca używa klucza prywatnego \( d \) i oblicza:
    \[
    M = C^d \mod n
    \]
\end{frame}


\begin{frame}{ECC}


Kryptografia krzywych eliptycznych (ECC) jest definiowana za pomocą następującego ogólnego równania w dwóch zmiennych z współczynnikami:

\[
    y^2 = x^3 + ax + b
\]

gdzie $a$ i $b$ są współczynnikami krzywej eliptycznej.

Delta krzywej eliptycznej jest określony jako:

\[
    \Delta = 4a^3 + 27b^2 \neq 0
\]

Warunek $\Delta \neq 0$ zapewnia, że krzywa tworzy \textbf{grupę algebraiczną}, co jest konieczne do zastosowania jej w kryptografii. Jeśli $\Delta = 0$, struktura matematyczna krzywej nie jest prawidłowa do użycia w szyfrowaniu.
\end{frame}

\begin{frame}{ECC}

    \textbf{Globalne elementy publiczne:}
    \begin{itemize}
        \item Wybór krzywej eliptycznej $E_q(a, b)$
        \item Wybór punktu bazowego $G(x, y)$ o dużym rzędzie $n$.
    \end{itemize}

    \textbf{Generowanie klucza Użytkownika A:}
    \begin{itemize}
        \item Użytkownik A wybiera klucz prywatny $V_A$, gdzie $V_A < n$.
        \item Oblicza klucz publiczny $P_A(x, y) = V_A \times G(x, y)$.
    \end{itemize}

    \textbf{Generowanie klucza Użytkownik B:}
    \begin{itemize}
        \item Użytkownik B wybiera klucz prywatny $V_B$, gdzie $V_B < n$.
        \item Oblicza klucz publiczny $P_B(x, y) = V_B \times G(x, y)$.
    \end{itemize}

\end{frame}


\begin{frame}{ECC}

    \begin{itemize}
        \item Użytkownik A wybiera wiadomość $P_m(x, y)$ oraz losową liczbę $k$, gdzie $1 < k < q$.
        \item Tworzy szyfrogram $C_m = ((k \times G(x, y)), (P_m(x, y) + k \times P_B(x, y)))$.
    \end{itemize}
    \begin{itemize}
        \item Użytkownik B otrzymuje szyfrogram $C_m$ jako $((x, y), (x', y'))$.
        \item Odszyfrowuje wiadomość:
        \[
            P_m(x, y) = (P_m(x, y) + k \times P_B(x, y)) - (k \times V_B \times G(x, y))
        \]
        \item Po wykonaniu operacji pozostaje oryginalna wiadomość $P_m(x, y)$.
    \end{itemize}


\end{frame}
