\section{Przyszłość}
\begin{frame}{Przyszłość w kontekście komputerów kwantowych}
       Algorytm Shore'a:
        \begin{itemize}
            \item Kwantowy algorytm rozkładu na czynniki
            \item Działa w czasie wielomianowym dla problemów faktoryzacji i logarytmu dyskretnego.
            \item Stosowany do RSA: rozkład liczby $N$ na czynniki pierwsze.
            \item Stosowany do ECC: rozwiązanie problemu logarytmu dyskretnego na krzywych eliptycznych.
        \end{itemize}
\end{frame}
\begin{frame}{Porównanie wymagań kwantowych}
    \begin{itemize}
        \item Liczba kubitów i operacji Toffoli dla RSA i ECC: \cite{quantum}
        \begin{itemize}
            \item RSA-3072 wymaga 6146 kubitów i $1.86 \times 10^{13}$ operacji Toffoli.
            \item ECC-P256 wymaga 2330 kubitów i $1.26 \times 10^{11}$ operacji Toffoli.
        \end{itemize}
        \item ECC wymaga mniej zasobów kwantowych niż RSA o równoważnym poziomie bezpieczeństwa.
    \end{itemize}
\end{frame}
\begin{frame}{Implikacje dla bezpieczeństwa}
    \begin{itemize}
        \item Komputery kwantowe będą mogły szybciej złamać ECC niż RSA.
        \item NIST(National Institute of Standards and Technology) pracuje nad postkwantową kryptografią (np. schematy oparte na kratkach - lattice-based cryptography).
        \item Organizacje powinny rozważyć migrację do algorytmów odpornych na komputery kwantowe.
    \end{itemize}
\end{frame}
\begin{frame}
    \frametitle{Dlaczego ECC jest bardziej podatne?}
    \begin{itemize}
        \item ECC bazuje na krótszych kluczach niż RSA (przy tej samej odporności klasycznej)
        \item ECC 256-bit odpowiada RSA 3072-bit pod względem klasycznej odporności.
        \item Algorytm Shore'a atakuje bezpośrednio strukturę matematyczną – mniejsze klucze ECC oznaczają, że wymagana liczba kubitów do ataku jest mniejsza
        \item W praktyce, ECC może zostać złamane szybciej niż RSA na komputerze kwantowym
    \end{itemize}
\end{frame}






% \begin{frame}{Bell's Theorem} % Opcjolanlne

% \end{frame}

% \begin{frame}{Regulacje prawne}
% \end{frame}
\begin{frame}{Regulacje prawne}
    \begin{itemize}
            \item Przepisy dotyczące szyfrowania różnią się w zależności od kraju.
            \item Kraje o restrykcyjnych przepisach, np. Chiny, mogą wymagać krajowych algorytmów.
            \item Wyzwania związane z międzynarodową zgodnością i polityką ochrony danych.
            \item Organizacje muszą dostosować swoje procedury do różnych regulacji.
    \end{itemize}
\end{frame}
\begin{frame}{Bezpieczeństwo a prywatność}
    \begin{itemize}
        \item Zrównoważenie ochrony prywatności z potrzebami organów ścigania.
        \item Wprowadzenie przepisów o backdoorach w systemach szyfrowania.
        \item Przepisy umożliwiające dostęp do zaszyfrowanych danych przez służby.
        \item Ryzyko naruszenia prywatności w imię bezpieczeństwa narodowego.
    \end{itemize}
\end{frame}
% \begin{frame}{Przyszłość kryptografii w kontekście komputerów kwantowych}
%     \begin{itemize}
%     \item RSA i ECC może zostać złamane przez komputery kwantowe.
%     \item Wymóg przejścia na post-kwantowe algorytmy szyfrowania.
%     \item Nowe algorytmy, takie jak lattice-based cryptography, mogą stać się standardem.
%     \item Regulacje mogą wymagać implementacji algorytmów odpornych na ataki kwantowe.
%     \end{itemize}
% \end{frame}
\begin{frame}{Podsumowanie}
    \begin{itemize}
        \item Komputery kwantowe mogą zagrażać bezpieczeństwu RSA i ECC.
        \item NIST pracuje nad algorytmami odpornymi na ataki kwantowe.
        \item Organizacje muszą dostosować swoje procedury do nowych wyzwań.
        \item Bezpieczeństwo i prywatność muszą być zrównoważone w regulacjach.
    \end{itemize}
\end{frame}
