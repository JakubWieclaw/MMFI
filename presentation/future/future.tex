\section{Przyszłość}
\begin{frame}{Przyszłość w kontekście komputerów kwantowych}
       Algorytm Shore'a:
        \begin{itemize}
            \item Kwantowy algorytm rozkładu na czynniki
            \item Działa w czasie wielomianowym dla problemów faktoryzacji i logarytmu dyskretnego.
            \item Stosowany do RSA: rozkład liczby $N$ na czynniki pierwsze.
            \item Stosowany do ECC: rozwiązanie problemu logarytmu dyskretnego na krzywych eliptycznych.
        \end{itemize}
\end{frame}
\begin{frame}{Porównanie wymagań kwantowych}
    \begin{itemize}
        \item Liczba kubitów i operacji Toffoli dla RSA i ECC:
        \begin{itemize}
            \item RSA-3072 wymaga 6146 kubitów i $1.86 \times 10^{13}$ operacji Toffoli.
            \item ECC-P256 wymaga 2330 kubitów i $1.26 \times 10^{11}$ operacji Toffoli.
        \end{itemize}
        \item ECC wymaga mniej zasobów kwantowych niż RSA o równoważnym poziomie bezpieczeństwa.
    \end{itemize}
\end{frame}
\begin{frame}{Implikacje dla bezpieczeństwa}
    \begin{itemize}
        \item Komputery kwantowe będą mogły szybciej złamać ECC niż RSA.
        \item NIST pracuje nad postkwantową kryptografią (np. schematy oparte na kratkach).
        \item Organizacje powinny rozważyć migrację do algorytmów odpornych na komputery kwantowe.
    \end{itemize}
\end{frame}
\begin{frame}
    \frametitle{Dlaczego ECC jest bardziej podatne?}
    \begin{itemize}
        \item ECC bazuje na krótszych kluczach niż RSA (przy tej samej odporności klasycznej)
        \item ECC 256-bit odpowiada RSA 3072-bit pod względem klasycznej odporności.
        \item Shor’s Algorithm atakuje bezpośrednio strukturę matematyczną – mniejsze klucze ECC oznaczają, że wymagana liczba kubitów do ataku jest mniejsza
        \item W praktyce, ECC może zostać złamane szybciej niż RSA na komputerze kwantowym
    \end{itemize}
\end{frame}






\begin{frame}{Bell's Theorem} % Opcjolanlne

\end{frame}

\begin{frame}{Regulacje prawne}

\end{frame}
% ---------------------
\begin{frame}[allowframebreaks, noframenumbering]{Bibliografia}
\printbibliography
\end{frame}
