\documentclass{beamer}
\usetheme{Berlin} % Bawcie się tym do woli
\usecolortheme{seahorse} % Bawcie się tym do woli
% Większość tych package'y to rzeczy które przeklejam z jakiejś starej templatki i pewnie nie wszystkie są potrzebne
\usepackage[polish]{babel}
\usepackage[utf8]{inputenc}
\usepackage[T1]{fontenc}
\usepackage{amsmath}
\usepackage{amssymb,amsfonts,amsthm}
\usepackage{multicol}
\usepackage{array}
\usepackage{geometry}
\usepackage{listings}
\usepackage{graphicx}
\usepackage{tabularx}
\usepackage{float}
\usepackage{hyperref}
% \usepackage{minted} %To chyba package do wklejania kodu
\title{Porównanie algorytmów kryptograficznych RSA i Krzywej Eliptycznej}
\author{Autorzy:\\ Krzysztof Dąrowski\\ Krzysztof Rudnik\\ Piotr Szczerba\\ Hussein Hazime\\ Jakub Więcław}
\date{}


\begin{document}

\begin{frame}
    \titlepage
\end{frame}

% Piotrek
\section{Wstęp}
\begin{frame}{Czym jest kryptografia asymetryczna}
    % Secure communications over insecure channels
\end{frame}

\begin{frame}{Jakie algortymy wybraliśmy}
    
\end{frame}
\begin{frame}{Pod jakimi kontami będziemy porównywać}
    
\end{frame}

\section{Sposoby działania wybranych algorytmów}
% Przedstawienie matematyki pod spodem i głównych idei wybranych algorytmów
\begin{frame}{Puzzle Merkle} % Krótka wspominka
    
\end{frame}
% ---------------------
% Hussein


\begin{frame}{RSA}
    
\end{frame}



\begin{frame}{ECC}
    
\end{frame}

% --------------------- 
% KUBA

% Od tego momentu porównujemy algorytmy między sobą
\section{Zastosowania}

\begin{frame}{Uwierzytelnianie}
    % Podpis
    % Certyfikat
\end{frame}


\begin{frame}{Kryptowaluty}
    
\end{frame}

\begin{frame}{Komunikatory E2E}
    
\end{frame}

\begin{frame}{IoT}
    % Urządzenia mają małą wydajność
    % Szyfrowanie komunikacji HTTP
\end{frame}

% ---------------------
% Krzysiek D

\section{Bezpieczeństwo}

\begin{frame}{Porówanie długości klucza}
    % Na tej samej długości klucza różne algorytmy dają różne bezpieczeństwo
\end{frame}

\begin{frame}{Problem wyboru krzywej ECC}
    % Są słabe krzywe, które można celowo wybrać jako backdoor
\end{frame}

\begin{frame}{RSA jest mocno przetestowane}
    % Są słabe krzywe, które można celowo wybrać jako backdoor
\end{frame}

% ---------------------
% Krzysiek R
\section{Przyszłość}
\begin{frame}{ECC jest łatwiejsze do złamania przez algorytm Shore'a niż RSA}
    
\end{frame}
\begin{frame}{Bell's Theorem} % Opcjolanlne
    
\end{frame}

\begin{frame}{Regulacje prawne}
    
\end{frame}

\end{document}