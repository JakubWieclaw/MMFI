\documentclass{beamer}
\usetheme{Berlin} % Bawcie się tym do woli
\usecolortheme{seahorse} % Bawcie się tym do woli
% Większość tych package'y to rzeczy które przeklejam z jakiejś starej templatki i pewnie nie wszystkie są potrzebne
\usepackage[polish]{babel}
\usepackage[utf8]{inputenc}
\usepackage[T1]{fontenc}
\usepackage{amsmath}
\usepackage{amssymb,amsfonts,amsthm}
\usepackage{multicol}
\usepackage{array}
\usepackage{geometry}
\usepackage{listings}
\usepackage{graphicx}
\usepackage{tabularx}
\usepackage{float}
\usepackage{hyperref}
\usepackage{mathspec} % wczytuje również fontspec
\setmainfont{Latin Modern Roman}
\setromanfont{Latin Modern Roman}
\setsansfont{Latin Modern Sans}
\setmonofont{Latin Modern Mono}
\setmathrm{Latin Modern Math}
\setmathfont(Digits,Latin)[Scale=MatchLowercase]{Latin Modern Math}
% \usepackage{minted} %To chyba package do wklejania kodu
\title{Porównanie algorytmów kryptograficznych RSA i Krzywej Eliptycznej}
\author{Autorzy:\\ Krzysztof Dąbrowski\\ Krzysztof Rudnik\\ Piotr Szczerba\\ Hussein Hazime\\ Jakub Więcław}
\date{2025-03-24}


\begin{document}

\begin{frame}
    \titlepage
\end{frame}

% Piotrek
\section{Wstęp}
\begin{frame}{Czym jest kryptografia asymetryczna}
    % Secure communications over insecure channels
\end{frame}

\begin{frame}{Jakie algortymy wybraliśmy}
    
\end{frame}
\begin{frame}{Pod jakimi kontami będziemy porównywać}
    
\end{frame}

\section{Sposoby działania wybranych algorytmów}
% Przedstawienie matematyki pod spodem i głównych idei wybranych algorytmów
\begin{frame}{Puzzle Merkle} % Krótka wspominka
    
\end{frame}
% ---------------------
% Hussein


\begin{frame}{RSA}
    
\end{frame}



\begin{frame}{ECC}
    
\end{frame}

% --------------------- 
% KUBA

% Od tego momentu porównujemy algorytmy między sobą
\section{Zastosowania}

\subsection{Uwierzytelnianie}
\begin{frame}{Uwierzytelnianie}
    \begin{itemize}
        \pause
        \item Podpisy cyfrowe
        \pause
        \item Certifikaty
    \end{itemize}
\end{frame}

\begin{frame}{Podpisy cyfrowe}
    
\end{frame}

\begin{frame}{Certifikaty}

\end{frame}
\subsection{}
% \subsection{Kryptowaluty}
\begin{frame}{Kryptowaluty}
\begin{center}
    \begin{tabular}{cc}
        \includegraphics[width=0.3\textwidth, height=0.3\textwidth ]{applications/graphics/Bitcoin.jpg} & \includegraphics[width=0.3\textwidth, height=0.3\textwidth]{applications/graphics/Ethereum.png} \\
        \includegraphics[width=0.3\textwidth, height=0.3\textwidth]{applications/graphics/Dogecoin.png} & \includegraphics[width=0.3\textwidth, height=0.3\textwidth]{applications/graphics/Litecoin.jpg} \\
    \end{tabular}
\end{center}
    
\end{frame}

% \subsection{Komunikatory E2E}
\begin{frame}{Komunikatory E2E}
    
\end{frame}

% \subsection{IoT}
\begin{frame}{IoT}
    % Urządzenia mają małą wydajność
    % Szyfrowanie komunikacji HTTP
\end{frame}

% ---------------------
% Krzysiek D

\section{Bezpieczeństwo}

\begin{frame}{Porówanie długości klucza}
    % Na tej samej długości klucza różne algorytmy dają różne bezpieczeństwo
\end{frame}

\begin{frame}{Problem wyboru krzywej ECC}
    % Są słabe krzywe, które można celowo wybrać jako backdoor
\end{frame}

\begin{frame}{RSA jest mocno przetestowane}
    % Są słabe krzywe, które można celowo wybrać jako backdoor
\end{frame}

% ---------------------
% Krzysiek R
\section{Przyszłość}
\begin{frame}{ECC jest łatwiejsze do złamania przez algorytm Shore'a niż RSA}
    
\end{frame}
\begin{frame}{Bell's Theorem} % Opcjolanlne
    
\end{frame}

\begin{frame}{Regulacje prawne}
    
\end{frame}

\end{document}