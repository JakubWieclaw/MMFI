% Większość tych package'y to rzeczy które przeklejam z jakiejś starej templatki i pewnie nie wszystkie są potrzebne
\usepackage[polish]{babel}
\usepackage[utf8]{inputenc}
\usepackage[T1]{fontenc}
\usepackage{amsmath}
\usepackage{amssymb,amsfonts,amsthm}
\usepackage{multicol}
\usepackage{array}
\usepackage{geometry}
\usepackage{listings}
\usepackage{graphicx}
\usepackage{tabularx}
\usepackage{float}
\usepackage{hyperref}
\usepackage{caption}
\usepackage[backend=biber,
        style=ieee,     % styl numeryczny IEEE prawie jak PN
        sorting=nyt,    % sortowanie spisu po nazwiskach
        citestyle=numeric-comp % kompaktowe odnośniki numeryczne typu [21-23]
    ]{biblatex}
\addbibresource{presentation.bib}
\DeclareNameAlias{default}{family-given} % nazwisko na początku
\usepackage{mathspec} % wczytuje również fontspec
\setmainfont{Latin Modern Roman}
\setromanfont{Latin Modern Roman}
\setsansfont{Latin Modern Sans}
\setmonofont{Latin Modern Mono}
\setmathrm{Latin Modern Math}
\setmathfont(Digits,Latin)[Scale=MatchLowercase]{Latin Modern Math}
% \usepackage{minted} %To chyba package do wklejania kodu
\usepackage{xcolor} % Pakiet do kolorowania tekstu
